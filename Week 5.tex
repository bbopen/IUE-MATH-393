\documentclass[a4paper,11pt]{article}
\usepackage{amsmath,amsthm,amssymb}
\usepackage{mathtools}
\DeclarePairedDelimiter\abs{\lvert}{\rvert}
\begin{document}
\newtheorem*{theorem1}{Theorem}
\newtheorem*{theorem2}{Theorem}
\newtheorem*{theorem3}{Theorem}
\newtheorem*{theorem4}{Theorem}
\newtheorem*{theorem5}{Theorem}
\newtheorem*{theorem6}{Theorem}
\newtheorem*{theorem7}{Theorem}
\newtheorem*{theorem8}{Theorem}
\title{MATH 393 Week 5 Assignment 2\textsuperscript{nd} Draft}
\author{Brett G. Bonner}
\date{February 24, 2014}
\maketitle
\linespread{1.5}
\newcounter{ProblemCounter}
\newcounter{SubsectionCounter}[ProblemCounter]
\addtocounter{ProblemCounter}{5} % set them to some other numbers than 0
\addtocounter{SubsectionCounter}{2} % same
%

\section*{\S 2.1 Exercise \arabic{ProblemCounter}: True or False?}
\textbf{\arabic{ProblemCounter}.\alph{SubsectionCounter}}
\( \varnothing \subseteq \{\varnothing, \{\varnothing\}\} \)\\
True - textbook theorem \textbf{2.1.1(a)} states ``For every set \(A, \varnothing \subseteq 
A\).'' Therefore \(\varnothing \subseteq \{\varnothing,\{\varnothing\}\}\).\\
\setcounter{SubsectionCounter}{4}

\noindent\textbf{\arabic{ProblemCounter}.\alph{SubsectionCounter}}
\( \{\varnothing\} \subseteq \{\varnothing, \{\varnothing\}\} \)\\
True - The set of an empty set, \(\{\varnothing\}\), is enumerated in the roster 
of set \(\{\varnothing, \{\varnothing\}\}\). Also, every element of \(\{\varnothing\}\) 
is an element of \(\{\varnothing, \{\varnothing\}\}\)\\
\setcounter{SubsectionCounter}{10}

\noindent\textbf{\arabic{ProblemCounter}.\alph{SubsectionCounter}}
\(\{1,2\} \in \{\{1,2,3\},\{1,3\},1,2\}\)\\
False - the set \{1,2\} is not a member of the righthand set.\\
\setcounter{SubsectionCounter}{12}

\noindent\textbf{\arabic{ProblemCounter}.\alph{SubsectionCounter}}
\(\{\{4\}\} \subseteq \{1,2,3,\{4\}\}\)\\
True - the set \{\{4\}\} is a subset of \{1,2,3,\{4\}\} as every element of \{\{4\}\} is also an element of \{1,2,3,\{4\}\}.\\
\setcounter{SubsectionCounter}{12}
\newpage

\setcounter{ProblemCounter}{8}
\setcounter{SubsectionCounter}{2}
\section*{\S 2.1 Exercise \arabic{ProblemCounter}: Prove part (c) of theorem 2.1.1: for all sets \(A, B,\) and \(C\), if \(A \subseteq B\) and \(B \subseteq C\), then \(A \subseteq C\).}

\begin{theorem2}
\( (\forall A)(\forall B)(\forall C)(A \subseteq B \wedge B \subseteq C) \Rightarrow (A \subseteq C) \)
\begin{proof}
Let \(A\), \(B\), \(C\) be any sets.\\
Suppose \(A \subseteq B\) and \(B \subseteq C\).\\
Let \(x\) be any object and element of \(A\).\\
By definition of a subject \(A \subseteq B \Leftrightarrow (\forall x)(x \in A \Rightarrow x \in 
B)\)\\
As \(x \in A\), the \(x \in B\).\\
\(B \subseteq C \subseteq (\forall x)(x \in B \Rightarrow x \in C)\).
As \(x \in B\), then \(x \in C\).\\
Therefore \(A \subseteq C \Leftrightarrow (\forall x) [(x \in A \Leftarrow x \in B)\wedge (x \in B \Leftarrow x \in 
C)]\)\\
As \(x \in A, x \in B, x \in C \therefore A \subseteq C\)\\
Because we supposed \(A, B, C\) are any sets and \(x\) is any object element of 
\(A\), we proved if \(A \subseteq B\) and \(B \subseteq C\), then \(A \subseteq 
C\).\end{proof}
\end{theorem2}

\newpage

\addtocounter{ProblemCounter}{6}
\addtocounter{SubsectionCounter}{1}
\section*{\S 2.1 Exercise \arabic{ProblemCounter}: Write the power set \(\mathcal{P}(X)\) for each of the following sets. }
\textbf{\arabic{ProblemCounter}.\alph{SubsectionCounter}}
\(X=\{0, \Delta, \Box\}\)\\
The definition of a power set \(\mathcal{P}(A)=\{B: B \subseteq A\}\)\\
\(\mathcal{P}(X)=\{\varnothing, \{0\}, \{\Delta\}, \{\Box\}, \{0, \Delta\}, \{0, \Box\}, \{\Delta, \Box\}, X \}\)
\newpage

\addtocounter{ProblemCounter}{2}
\setcounter{SubsectionCounter}{2}
\section*{\S 2.1 Exercise \arabic{ProblemCounter}: List all the proper subsets for each of the following sets. }
\textbf{\arabic{ProblemCounter}.\alph{SubsectionCounter}}
\(\{\varnothing, \{\varnothing\}\}\)\\
\(\mathcal{P}(\{\varnothing, \{\varnothing\}\} = \{\varnothing, \{\varnothing\}, \{\{\varnothing\}\}, 
\{\varnothing,\{\varnothing\}\}\}\)\\
The number of subset elements of a powerset is \(2^2 = 4\)\\
if \(P \subset \mathcal{P}(\{\varnothing, \{\varnothing\}\})\)\\
therefore the proper subsets of \(P = \{\varnothing, \{\varnothing\}, \{\{\varnothing\}\}\}\)\\

\newpage

\setcounter{ProblemCounter}{7}
\setcounter{SubsectionCounter}{1}
\section*{\S 2.2 Exercise \arabic{ProblemCounter}: Prove the remaining parts of theorem 2.2.1 }
\textbf{\arabic{ProblemCounter}.\alph{SubsectionCounter}}
For all sets \(A, B, \) and \(C, A \subseteq A \cup B\)
\begin{theorem4}
\(A \subseteq A \cup B\), for all sets \(A, B\)\\
Semi-formally \((\forall A)(\forall B)(A \subseteq A \Rightarrow A \cup B)\)
\begin{proof}
Suppose \(A\) and \(B\) are sets.\\
Let \(x\) be any object element of \(A (x \in A)\)\\
As \(x \in A\), by the definition of union operation...\\
\(A \cup B = \{x: x \in A or \in B\}, (x \in A) \Rightarrow (x \in A \wedge x \in 
B)\)\\
As \(x \in A\), it is a tautology \(P \Rightarrow (P \vee Q)\therefore x \in A \cup 
B\)\\
Because we let \(x \in A\), we proved \(x \in A \cup B\) and \(A \subseteq A\),\\
therefore \(A \subseteq (A \cup B)\)
\end{proof}
\end{theorem4}

\addtocounter{SubsectionCounter}{4}
\noindent\textbf{\arabic{ProblemCounter}.\alph{SubsectionCounter}}
\(A \cap A = A\)

\begin{theorem4}
\(A \cap A = A\) for all sets A.
\begin{proof}
Let \(A\) be any set. Suppose x is any object element of \(A, x \in A\).\\
By definition of intersection, \(A \cap A = \{x: x \in A\) and \(x \in A\}\).\\
As \(x \in A, x \in A \cap A\) by definition of intersection.\\
As \(A=A\) by definition of equality and because we let \(x \in A\), we proved \(x \in A \cap A\) and 
therefore \(A \cap A = A\).
\end{proof}
\end{theorem4}

\newpage

\setcounter{ProblemCounter}{7}
\setcounter{SubsectionCounter}{7}
\section*{\S 2.2 Exercise \arabic{ProblemCounter}: Prove the remaining parts of theorem 2.2.1 }
\textbf{\arabic{ProblemCounter}.\alph{SubsectionCounter}}
For all sets \(A, B\) and \(C\)\\
\(A \cup B = B \cup A\)
\begin{theorem4}
\(A \cup B = B \cup A\) for all sets A, B
\begin{proof}
Let \(A\) and \(B\) be any sets. Suppose \(x\) is any object.\\
By definition of union, \(A \cup B = \{x: x \in A\) or \(x \in B\}\).
\begin{align*}
x \in A \cup B \text{ iff } x \in A \text{ or } x \in B\\
\text{iff } x \in B \text{ or } x \in A\\
\end{align*}
By definition of union, \(B \cup A = \{x: x \in B \text{ or } x \in A\}\).
\begin{align*}
x \in B \cup A \text{ iff } x \in B \text{ or } x \in A\\
\text{iff } x \in A \text{ or } x \in B\\
\end{align*}
By definition of equivalence,
\(A \cup B = B \cup A \text{ iff } (\forall x)(x \in A \cup B \Leftrightarrow x \in B \cup 
A)\)
Let \(x \in A\), therefore \(x \in A \cup B \text{ and } x \in B \cup A\) and 
therefore \(A \cup B = B \cup A\).\\
Let \(x \in B\), therefore \(x \in B \cup A \text{ and } x \in A \cup B\) and 
therefore \(A \cup B = B \cup A\).
By definition of union and equivalence, we showed:
\begin{align*}
\text{if } x \in A \text{ or } x \in B \text{ or } (x \in A \text{ and } x \in B), A \cup B = B \cup A 
\end{align*}
\end{proof}
\end{theorem4}

\newpage

\setcounter{ProblemCounter}{7}
\setcounter{SubsectionCounter}{15}
\section*{\S 2.2 Exercise \arabic{ProblemCounter}: Prove the remaining parts of theorem 2.2.1 }
\textbf{\arabic{ProblemCounter}.\alph{SubsectionCounter}}
For all sets \(A, B\) and \(C\)\\
\(A \subseteq B\) iff \(A \cup B = B\)
\begin{theorem5}
\(A \subseteq B\) iff \(A \cup B = B\) for all sets \(A\) and \(B\)
\begin{proof}
Assume that \(A \subseteq B\) and \(x\) is arbitrary.\\
To show \(A \cup B = B\), we must show by definition of equivalence that\\
\(A \cup B \subseteq B \text{ and } B \subseteq A \cup B\).\\\\
For \(A \cup B \subseteq B:\)\\
Let \(x \in A \cup B\). By definition of union, \(x \in A\) or \(x \in B\).\\
As we supposed \(A \subseteq B\), if \(x \in A\) then \(x \in B\) as well.\\
Therefore, if \(x \in A \cup B\), then \(x \in B\) and therefore \(A \cup B \subseteq 
B\)\\

\noindent For \(B \subseteq A \cup B:\)\\
Let \(x \in B\), therefore \(x \in A \cup B\) by definition of union and therefore \(B \subseteq A \cup B\)\\

\noindent It is demonstrated if \(A \subseteq B\) then \(A \cup B \subseteq B\)
and \(B \subseteq A \cup B\), then \(A \cup B = B\).\\
To prove if \(A \cup B = B\) then \(A \subseteq B\), assume \(A \cup B = B\).\\
\(A \subseteq B \Leftrightarrow (\forall x)(x \in A \Rightarrow x \in B)\).\\
Let \(x \in A\), as \(A \cup B = B\), \(x \in A\) or \(x \in B\), thus \(x \in A \cup 
B\).\\
If \(x \in A\), the \(x \in B\) and therefore \(A \subseteq B\) and therefore we 
proved\\ \(A \subseteq B\) iff \(A \cup B = B\)
\end{proof}
\end{theorem5}

\newpage

\setcounter{ProblemCounter}{8}
\setcounter{SubsectionCounter}{2}
\section*{\S 2.2 Exercise \arabic{ProblemCounter}: Let \(U\) be the universe, and let \(A\) and \(B\) be subsets of \(U\), then: }
\textbf{\arabic{ProblemCounter}.\alph{SubsectionCounter}}
\(A \cup A^c = U\)
\begin{theorem5}
\(A \cup A^c = U\)
\begin{proof}
Let \(U\) be the universe and let \(A\) be a subset of \(U\).\\
Let \(x \in A \cup A^c\). By definition of union, \(A \cup A^c = \{x: x \in A \text{ or } x \in A^c 
\}\).\\
Therefore, \(x \in A\) or \(x \in A^c\).\\
As \(A \subseteq U\) and \(A^c \subseteq U\), \(x \in U\) therefore \(A \cup A^c \subseteq 
U\).\\
Suppose \(x \in U\), then \(x \in A\) or \(x \notin A\).\\
As \(x \in A \text{ or } x \notin A\), \(x \in A \cup A^c\).\\
Therefore \(U \subseteq A \cup A^c\) by the theorem \(A \subseteq A\).\\
Therefore, as we let \(A\) be a subset of \(U\) and let \(x \in A \cup A^c\), we 
proved \(A \cup A^c = U\).
\end{proof}
\end{theorem5}

\newpage

\setcounter{ProblemCounter}{8}
\setcounter{SubsectionCounter}{4}
\section*{\S 2.2 Exercise \arabic{ProblemCounter}: Let \(U\) be the universe, and let \(A\) and \(B\) be subsets of \(U\), then: }
\textbf{\arabic{ProblemCounter}.\alph{SubsectionCounter}}
\(A - B = A \cap B^c\)
\begin{theorem5}
\(A - B = A \cap B^c\)
\begin{proof}
By definition of set equivalence, we must prove \(A - B \subseteq A \cap B^c\) 
and \(A \cap B^c \subseteq A - B\) to prove \(A - B = A \cap B^c\).\\\\
For \(A - B \subseteq A \cap B^c:\)\\
Assume \(A\) and \(B\) are sets and let \(x \in A - B\).\\
By definition of difference, \(x \in A\) and \(x \notin B\).\\
Then by definition of set complement, \(x \in A\) and \(x \in B^c\).\\ 
Therefore, \(x \in A \cap B^c\) therefore \(A - B \subseteq A \cap B^c\)\\

\noindent For \(A \cap B^c \subseteq A - B\):\\
Assume \(A\) and \(B\) are sets and let \(x \in A \cap B^c\).\\
Then \(x \in A\) and \(x \in B^c\), then \(x \notin B\), therefore \(x \in A - B\).\\
Therefore \(A \cap B^c \subseteq A - B\).\\

\noindent As \(A - B \subseteq A \cap B^c\) and \(A \cap B^c \subseteq A - B\), by 
definition of equivalence, \(A - B = A \cap B^c\)
\end{proof}
\end{theorem5}

\newpage

\setcounter{ProblemCounter}{8}
\setcounter{SubsectionCounter}{7}
\section*{\S 2.2 Exercise \arabic{ProblemCounter}: Let \(U\) be the universe, and let \(A\) and \(B\) be subsets of \(U\), then: }
\textbf{\arabic{ProblemCounter}.\alph{SubsectionCounter}}
\((A \cap B)^c = A^c \cup B^c\)
\begin{theorem5}
\((A \cap B)^c = A^c \cup B^c\)
\begin{proof}
By definition of set equivalence, we must prove \((A \cap B)^c \subseteq A^c \cup B^c\) 
and that \(A^c \cup B^c \subseteq (A \cap B)^c\) in order for  \((A \cap B)^c = A^c \cup B^c\).\\\\
For \((A \cap B)^c \subseteq A^c \cup B^c\):\\
Let \(x \in (A \cap B)^c\), where \(A\) and \(B\) are subsets in universe 
\(U\).\\
Then \(x \in U - (A \cap B)\) and therefore \(x \notin A \cap B\).\\
Therefore \(\neg (x \in A \cap B)\) by DeMorgan's Law \(\neg (x \in A \text{ and } x \in 
B)\), thus \(x \in A^c\) or \(x \in B^c\).\\
Then \(x \in A^c \cup B^c\) and therefore \((A \cup B)^c \subseteq A^c \cup 
B^c\)\\

\noindent For \(A^c \cup B^c \subseteq (A \cap B)^c\):\\
Let \(x \in A^c \cup B^c\).\\
Then \(x \in U - A\) or \(x \in U - B\) and therefore \(x \in A\) or \(x \notin 
B\).\\
Therefore by DeMorgan's Law \(\neg (x \in A\) and \(x \in B)\).\\
Therefore \(x \in (A \cap B)^c\),\\
and therefore \(A^c \cup B^c \subseteq (A \cap B)^c\)\\

\noindent Thus as \((A \cap B)^c \subseteq A^c \cup B^c\) and \(A^c \cup B^c \subseteq (A \cup 
B)^c\),\\ we proved \((A \cap B)^c = A^c \cup B^c\)
\end{proof}
\end{theorem5}

\newpage

\setcounter{ProblemCounter}{8}
\setcounter{SubsectionCounter}{8}
\section*{\S 2.2 Exercise \arabic{ProblemCounter}: Let \(U\) be the universe, and let \(A\) and \(B\) be subsets of \(U\), then: }
\textbf{\arabic{ProblemCounter}.\alph{SubsectionCounter}}
\(A \cap B = \varnothing\) iff \(A \subseteq B^c\)
\begin{theorem5}
\(A \cap B = \varnothing\) iff \(A \subseteq B^c\)
\begin{proof}
We affirm the biconditional by conjunction of converses:\\
i) Suppose \(A\) and \(B\) are subsets of \(U\), \(A \cap B = \varnothing\), and 
let \(x \in A\).
As \(A \cap B = \varnothing, x \in B\), therefore \(x \in B^c\)\\
Therefore if \(A \cap B \neq 0\), then \(A \subseteq B^c\)\\

\noindent ii) Suppose \(A\) and \(B\) are subsets of \(U\), suppose \(A \subseteq 
B^c\), and let \(x \in A \cap B\).\\
By definition of intersection, \(x \in A\) and \(x \in B\).\\
But since \(A \subseteq B^c\), \(x \in A\) and therefore \(x \in B^c\), \(x \notin B\) 
which is a contradiction to \(x \in B\) from \(x \in A \cap B\).\\
Therefore if \(A \subseteq B^c\) then \(A \cap B = \varnothing\).\\

\noindent Therefore, \(A \cap B = \varnothing\) iff \(A \subseteq B^c\)
\end{proof}
\end{theorem5}

\newpage
\setcounter{ProblemCounter}{15}
\setcounter{SubsectionCounter}{1}
\section*{\S 2.2 Exercise \arabic{ProblemCounter}: If \(A, B, C, D\) are sets, then: }
\textbf{\arabic{ProblemCounter}.\alph{SubsectionCounter}}
\(A \times {(B \cap C)}={(A \times B)}\cap{(A \times C)}\)
\begin{theorem5}
\(A \times {(B \cap C)}={(A \times B)}\cap{(A \times C)}\)
\begin{proof}
Let \(A, B, C\) be sets. We must show that for an ordered pair\\
\({(a,b)},{(a,b)} \in A \times {(B \cap C)}\) iff \({(a,b)} \in {(A \times B) \cap {(A \times 
C)}}\)
\begin{align*}
  {(a,b)} \in A \times {(B \cap C)}&\\
  &\text{iff } a \in A \text{ and } b \in B \cap C\\
  &\text{iff } a \in A \text{ and } b \in B \text{ and } b \in C\\
  &\text{iff } a \in A \text{ and } b \in B \text{ and } a \in A \text{ and } b \in C\\
  &\text{iff } {(a,b)} \in A \times B \text{ and } {(a,b)} \in A \times C\\
  &\text{iff } {(a,b)} \in {(A \times B)} \cap {(A \times C)}
\end{align*}
\end{proof}
\end{theorem5}

\setcounter{SubsectionCounter}{3}
\noindent\textbf{\arabic{ProblemCounter}.\alph{SubsectionCounter}}
\({(A \times B)} \cap {(C \times D)} = {(A \cap C)} \times {(B \cap D)}\)
\begin{theorem5}
\({(A \times B)} \cap {(C \times D)} = {(A \cap C)} \times {(B \cap D)}\)
\begin{proof}
We suppose \(A, B, C, D\) are sets. We must show that for an ordered pair\\
\({(x,y)},{(x,y)} \in {(A \times B)} \cap {(C \times D)}\) iff \({(x,y) \in {(A \cap C)} \times {(B \cap D)}}\)
\begin{align*}
  {(x,y)} \in {(A \times B)} \cap {(C \times D)}&\\
  &\text{iff } {(x,y)} \in {(A \times B)} \text{ and } {(x,y)} \in {(C \times D)}\\
  &\text{iff } {(x \in A \text{ and } y \in B)} \text{ and } {(x \in C \text{ and } y \in D)}\\
  &\text{iff } {(x \in A \text{ and } x \in C)} \text{ and } {(y \in B \text{ and } y \in D)}\\
  &\text{iff } {x \in {(A \cap C)} \text{ and } y \in {(B \cap D)}}\\
  &\text{iff } {(x,y)} \in {(A \cap C)} \times {(B \cap D)}
\end{align*}
\end{proof}
\end{theorem5}

\end{document}