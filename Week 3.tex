\documentclass[a4paper,11pt]{article}
\usepackage{amsmath,amsthm,amssymb}
\usepackage{mathtools}
\DeclarePairedDelimiter\abs{\lvert}{\rvert}
\begin{document}
\newtheorem*{theorem1}{Theorem}
\newtheorem*{theorem2}{Theorem}
\newtheorem*{theorem3}{Theorem}
\newtheorem*{theorem4}{Theorem}
\newtheorem*{theorem5}{Theorem}
\newtheorem*{theorem6}{Theorem}
\newtheorem*{theorem7}{Theorem}
\newtheorem*{theorem8}{Theorem}
\title{MATH 303 Week 3 Assignment 2\textsuperscript{nd} Draft}
\author{Brett G. Bonner}
\date{February 10, 2014}
\maketitle
\linespread{1.5}
\newcounter{ProblemCounter}
\newcounter{SubsectionCounter}[ProblemCounter]
\addtocounter{ProblemCounter}{6} % set them to some other numbers than 0
\addtocounter{SubsectionCounter}{1} % same
%

\section*{\S 1.4 Exercise \arabic{ProblemCounter}: Let \( a\) and \( b \) be real numbers. Prove that}
\textbf{\arabic{ProblemCounter}.\alph{SubsectionCounter}}
\( |ab| = |a||b|\)
\begin{theorem1}
\( |ab| = |a||b|\)
\begin{proof}
Let \(a\) and \(b\) be real numbers. Suppose for any real number \(x\), the 
absolute value of x, \(|x|\), is a piecewise function defined as:
\begin{align}
  \label{quad}
     |x| = \left\{
     \begin{array}{lr}
       -x, & x < 0\\
        x, & x \geq 0
     \end{array}
   \right.
\end{align}
For \(a\) and \(b\), we show that \(|ab| = |a||b|\) for all four possible 
cases:\\ \\
\text{\underline{Case 1:} }\(a \geq 0 \text{, and } b \geq 0\)
\begin{gather*}
\label{case1}
|a| = a \text{, and } |b|=b\\
\text{As \(a \geq 0\) and \(b \geq 0\), \(ab \geq 0 \therefore |ab|=ab\)}\\
\text{As }
|ab| = ab\\
ab=(a)(b)\\
(a)(b)=|a||b|
\end{gather*}

\noindent\text{\underline{Case 2:} }\(a \geq 0 \text{, and } b < 0\)
\begin{gather*}
\label{case2}
|a| = a \text{, and } |b|=-b\\
\text{As \(a \geq 0\) and \(b < 0\), \(ab \leq 0\).}\\
\text{As } ab \leq 0, |ab|=-ab\\
|ab|= -ab\\
-ab = (a)(-b)\\
(a)(-b) = |a||b|
\end{gather*}

\noindent\text{\underline{Case 3:} }\(a < 0 \text{, and } b \geq 0\)
\begin{gather*}
\label{case3}
|a| = -a \text{, and } |b|=b\\
\text{As \(a < 0\) and \(b \geq 0\), \(ab \leq 0\).}\\
\text{As } ab \leq 0, |ab|=-ab\\
|ab|= -ab\\
-ab = (-a)(b)\\
(-a)(b) = |a||b|
\end{gather*}

\noindent\text{\underline{Case 4:} }\(a < 0 \text{, and } b < 0\)
\begin{gather*}
\label{case4}
|a| = -a \text{, and } |b|=-b\\
\text{As \(a < 0\) and \(b < 0\), \(ab > 0\).}\\
\text{As } ab > 0, |ab|=ab\\
|ab|= ab\\
ab = (a)(b)\\
(a)(b) = |a||b|
\end{gather*}
Therefore, \(|ab|=|a||b|\) for all possible cases.\\
We proved if \(a\) and \(b\) are real numbers, then \(|ab|=|a||b|\)
\end{proof}
\end{theorem1}
\newpage

\addtocounter{SubsectionCounter}{2}
\section*{\S 1.4 Exercise \arabic{ProblemCounter}: Let \( a\) and \( b \) be real numbers. Prove that}
\textbf{\arabic{ProblemCounter}.\alph{SubsectionCounter}}
\(|\frac{a}{b}| = \frac{|a|}{|b|}\), for \(b \neq 0\)

\begin{theorem2}
\(|\frac{a}{b}| = \frac{|a|}{|b|}\), for \(b \neq 0\)
\begin{proof}
Let \(a\) and \(b\) be real numbers. Suppose for any real number \(x\), the 
absolute value of x, \(|x|\), is a piecewise function defined as:
\begin{align}
  \label{quad}
     |x| = \left\{
     \begin{array}{lr}
       -x, & x < 0\\
        x, & x \geq 0
     \end{array}
   \right.
\end{align}
For \(a\) and \(b\), we show that \(|\frac{a}{b}| = \frac{|a|}{|b|}\), for \(b \neq 0\) for all four possible 
cases:\\ \\
\text{\underline{Case 1:} }\(a \geq 0 \text{, and } b > 0\)
\begin{gather*}
\label{case1}
|a| = a \text{, and } |b|=b\\
\text{As } a \geq 0 \text{ and } b > 0, \frac{a}{b} \geq 0 \therefore \abs*{\frac{a}{b}} = 
\frac{a}{b}\\
\text{As } \abs*{\frac{a}{b}} = \frac{a}{b} = a \times \frac{1}{b} = \abs*{a}\times\frac{1}{\abs{b}} = \frac{\abs{a}}{\abs{b}}
\end{gather*}

\noindent \text{\underline{Case 2:} }\(a < 0 \text{, and } b > 0\)
\begin{gather*}
\label{case2}
|a| = -a \text{, and } |b|=b\\
\text{As } a < 0 \text{ and } b > 0, \frac{a}{b} < 0 \therefore \abs*{\frac{a}{b}} = -\frac{a}{b}\\
\text{As } \abs*{\frac{a}{b}} = -\frac{a}{b} = (-a) \times \frac{1}{b} = \abs*{a}\times\frac{1}{\abs{b}} = \frac{\abs{a}}{\abs{b}}
\end{gather*}

\noindent\text{\underline{Case 3:} }\(a \geq 0 \text{, and } b < 0\)
\begin{gather*}
\label{case3}
|a| = a \text{, and } |b|=-b\\
\text{As } a \geq 0 \text{ and } b < 0, \frac{a}{b} \leq 0 \therefore \abs*{\frac{a}{b}} = -\frac{a}{b}\\
\text{As } \abs*{\frac{a}{b}} = -\frac{a}{b} = (a) \times \frac{1}{-b} = \abs*{a}\times\frac{1}{\abs{b}} = \frac{\abs{a}}{\abs{b}}
\end{gather*}

\noindent\text{\underline{Case 4:} }\(a < 0 \text{, and } b < 0\)
\begin{gather*}
\label{case4}
|a| = -a \text{, and } |b|=-b\\
\text{As } a < 0 \text{ and } b < 0, \frac{a}{b} > 0 \therefore \abs*{\frac{a}{b}} = \frac{a}{b}\\
\text{As } \abs*{\frac{a}{b}} = \frac{a}{b} = (a) \times \frac{1}{b} = \abs*{a}\times\frac{1}{\abs{b}} = \frac{\abs{a}}{\abs{b}}
\end{gather*}
Therefore, \(\abs*{\frac{a}{b}}=\frac{\abs*{a}}{\abs*{b}}\) for \(b \neq 0\) for all possible cases.\\
We proved if \(a\) and \(b\) are real numbers, then \(\abs*{\frac{a}{b}}=\frac{\abs*{a}}{\abs*{b}}\) for \(b \neq 0\)
\end{proof}
\end{theorem2}

\newpage
\addtocounter{ProblemCounter}{1}
\addtocounter{SubsectionCounter}{5}
\section*{\S 1.4 Exercise \arabic{ProblemCounter}: Suppose \( a\), \( b\), \(c\), and \(d\) are integers. Prove that}
\textbf{\arabic{ProblemCounter}.\alph{SubsectionCounter}}
if \(a\) divides \(b\), then \(a\) divides \(bc\)

\begin{theorem3}
If \(a\) divides \(b\), then \(a\) divides \(bc\)
\begin{proof}
Let \(a\), \(b\), and \(c\) be integers.\\
We define ``divides'' as the condition where if the ratio of some integers \(x\) and \(y\) is an integer \(z\), then \(y\) is said to divide \(x\).\\
Assume \(a\) divides \(b\).\\
By definition, there exists an integer \(k\) such that \(b=ak\).\\
Therefore, \(bc = (ak)c = a(kc)\), where \(a\) divides \(bc\).\\
We proved if \(a\) divides \(b\), then \(a\) divides \(bc\).
\end{proof}
\end{theorem3}

\vspace{\baselineskip}
\addtocounter{SubsectionCounter}{3}
\noindent\textbf{\arabic{ProblemCounter}.\alph{SubsectionCounter}}
if \(a\) divides \(b\) and \(c\) divides \(d\), then \(ac\) divides \(bd\)

\begin{theorem4}
If \(a\) divides \(b\) and \(c\) divides \(d\), then \(ac\) divides \(bd\).
\begin{proof}
Let \(a\), \(b\), \(c\), and \(d\) be integers.\\
We define ``divides'' as the condition where if the ratio of some integers \(x\) and \(y\) is an integer \(z\), then \(y\) is said to divide \(x\).\\
Assume \(a\) divides \(b\) and \(c\) divides \(d\).\\
By definition, there exists integers \(k\) and \(m\) such that \(b=ak\) and \(d=cm\).\\
\(bd = (ak)(cm) = (ac)(km)\)\\
Therefore, \(ac\) divides \(bd\).\\
We proved if \(a\) divides \(b\) and \(c\) divides \(d\), then \(ac\) divides \(bd\).
\end{proof}
\end{theorem4}
\newpage

\setcounter{ProblemCounter}{3}
\setcounter{SubsectionCounter}{3}
\section*{\S 1.5 Exercise \arabic{ProblemCounter}: Let \( x\), \( y\), and \(z\) be integers. Write a proof by contraposition to show that }
\textbf{\arabic{ProblemCounter}.\alph{SubsectionCounter}}
if \(x^2\) is not divisible by 4, then \(x\) is odd.

\begin{theorem5}
If \(x^2\) is not divisible by 4, then \(x\) is odd for all integers \(x\).
\begin{proof}
Let \(x\) be an integer, and assume \(x\) is not odd.\\
As \(x\) is not odd, the parity of \(x\) is even.\\
`Even' is defined as an integer of the form \(n = 2k\), where \(k\) is an 
integer and \(n\) is an integer.\\
As \(x\) is even, \(x=2y\) for an integer \(y\)
\begin{gather*}
x^2 = {(2y)}^2 = 4y^2
\end{gather*}
By definition, \(4\) divides \(x^2\) thus if \(x\) is not odd, then \(x^2\) is 
divisible by \(4\).\\
Therefore, if \(x^2\) is not divisible by \(4\), then \(x\) is odd.\\
By contraposition, it is proved that if \(x^2\) is not divisible by \(4\), then \(x\) 
is odd.
\end{proof}
\end{theorem5}
\setcounter{SubsectionCounter}{3}
\noindent\textbf{\arabic{ProblemCounter}.\alph{SubsectionCounter}}
if \(xy\) is even, then \(x\) or \(y\) is even.

\begin{theorem6}
If \(xy\) is even, then \(x\) or \(y\) is even.
\begin{proof}
Let \(x\) and \(y\) be integers, and assume neither \(x\) nor \(y\) is even.
As neither \(x\) nor \(y\) are even, \(x\) and \(y\) are odd.
`Odd' is defined as an integer of the form \(n = 2k+1\), where \(k\) is an 
integer and \(n\) is an integer.
As \(x\) and \(y\) are odd, \(x=2m+1\) and \(y=2p+1\), where \(m\) and \(p\) are 
integers.
\begin{equation*}
xy=(2m+1)(2p+1)=4mp+2p+2m+1=2{(2mp+p+m)}+1
\end{equation*}
Let an integer \(z=2mp+p+m\), therefore \(2z+1=xy\), proving the product of 
\(x\) and \(y\) is odd. Therefore if \(xy\) is even, \(x\) or \(y\) is even.\\
We proved by contraposition that if \(xy\) is even, then either \(x\) or \(y\) 
is even.
\end{proof}
\end{theorem6}

\newpage
\setcounter{ProblemCounter}{5}
\section*{\S 1.5 Exercise \arabic{ProblemCounter}: A circle has center {(2,4)}.}
\setcounter{SubsectionCounter}{1}
\noindent\textbf{\arabic{ProblemCounter}.\alph{SubsectionCounter}}
Prove that {(-1,5)} and {(5,1)} are not both on the same circle.
\begin{theorem6}
  If a circle has center {(2,4)}, {(-1,5)} and {(5,1)} are not both on the circle.
  \begin{proof}
    Assume {(-1,5)} and {(5,1)} are both on the circle with center {(2,4)}. A circle 
    with radius \(r\) and center \((a,b)\) is the set of all points \((x,y)\) 
    such that \({(x-a)}^2+{(y-b)}^2=r^2\).\\
\noindent\text{\underline{Case 1:} the radius of the circle with center {(2,4)} and point on circle 
(-1,5)}
\begin{multline*}
\label{Ex5case1}
r^2={(-1-2)}^2+{(5-4)}^2\\
=9+1\\
=10
\end{multline*}
\noindent\text{\underline{Case 2:} the radius of the circle with center (2,4) and point on circle 
(5,1)}
\begin{multline*}
r^2={(5-2)}^2+{(1-4)}^2\\
=9+9\\
=18
\end{multline*}  
As {(-1,5)} and {(5,1)} is on the same circle, the radius \(r\) from the center to 
each point should be equal. As the radius in case 1 is not equivalent with the radius in case 2, 
there is a contradiction. Therefore, {(-1,5)} and {(5,1)} are not both on a circle 
with center (2,4).
\end{proof}
\end{theorem6}
\newpage
\section*{\S 1.5 Exercise \arabic{ProblemCounter}: A circle has center {(2,4)}.}
\setcounter{SubsectionCounter}{2}
\noindent\textbf{\arabic{ProblemCounter}.\alph{SubsectionCounter}}
 Prove that if the radius is less than 5, then the circle does not intersect the line 
\(y=x-6\).

\begin{theorem8}
If a circle with center {(2,4)} has a radius less than 5, then the circle does not 
intersect with the line \(y=x-6\).
\begin{proof}
Assume a circle intersects with the line \(y=x-6\) and the circle has radius \(r < 
5\).\\
Let the point {(a,b)} be a point of intersection of the line and circle, then 
\(b=a-6\).\\
A circle with radius \(<\) 5, center (2,4), and point (a,b) is written:
\begin{multline*}
  {(a-2)}^2 + {(b-4)}^2 < 5^2\\
  [{(a-2)}^2 + {(b-4)}^2 < 5^2] = [{(a-2)}^2 + {(a-6-4)}^2 < 25] =\\
  {(2a^2 - 24a + 104 < 25)} = {(2a^2 - 24a + 79 < 0)}
\end{multline*}
To determine if the quadratic inequality \(2a^2 - 24a + 79 < 0\) has a solution, the 
discriminant (\(\Delta\)) of the quadratic equation is evaluated:
\begin{equation*}
  \Delta = -24^2 - 4(2)(79) = -56
\end{equation*}
As \(\Delta < 0\), by definition of the discriminant, the quadratic equation 
\[2a^2 - 24a + 79 = 0\] has no real-number solutions, meaning the expression \[2a^2 - 24a + 79\] is either \textit{always} positive or \textit{always} negative. Testing with \(a=0\) will determine if the solution is either positive or negative:\\
\begin{multline*}
  2{(0)}^2-24{(0)}+79\\
  =0+0+79\\
  =79
\end{multline*}
As testing with \(a=0\) indicates the quadratic expression is positive, a contradiction exists with the equation of the circle intersecting with line \(y=x-6\) at point (a,b), expressed by \(2a^2 - 24a + 79 < 0\), which indicates the solution to the expression should be negative.\\
Therefore, a contradiction exists where the circle can not have a radius less than 5 \textit{and} intersect with the line \(y=x-6\).\\
It is proved by contradiction that if a circle with center {(2,4)} has a radius 
less than 5, then the circle does not intersect with the line \(y=x-6\)
\end{proof}
\end{theorem8}



\end{document}