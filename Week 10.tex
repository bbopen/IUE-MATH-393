\documentclass[a4paper,11pt]{article}
\usepackage{amsmath,amsthm,amssymb}
\usepackage{mathtools}
\usepackage{mathrsfs}
\usepackage{setspace}
\usepackage{enumerate}
\usepackage{caption}
\DeclarePairedDelimiter\abs{\lvert}{\rvert}
\PassOptionsToPackage{usenames,dvipsnames,svgnames}{xcolor}  
\usepackage{tikz}
\usepackage{cancel}
\usetikzlibrary{arrows,positioning,automata,fit,shapes,backgrounds}
\usepackage{framed}
\begin{document}
\newtheorem*{theorem1}{Theorem}
\newtheorem*{theorem2}{Theorem}
\newtheorem*{theorem3}{Theorem}
\newtheorem*{theorem4}{Theorem}
\newtheorem*{theorem5}{Theorem}
\newtheorem*{theorem6}{Theorem}
\newtheorem*{theorem7}{Theorem}
\newtheorem*{theorem8}{Theorem}
\newtheorem*{theorem9}{True/False?}
\title{MATH 393 Week 10 Assignment 2\textsuperscript{nd} Draft}
\author{Brett Bonner}
\date{March 28, 2014}
\maketitle
\doublespacing
\newcounter{ProblemCounter}
\newcounter{SubsectionCounter}[ProblemCounter]
\addtocounter{ProblemCounter}{6} % set them to some other numbers than 0
\addtocounter{SubsectionCounter}{3} % same
%

\setcounter{ProblemCounter}{2}
\section*{\S 3.4 Exercise \arabic{ProblemCounter}: Let \(A=\{a,b,c\}\). Give an example of a relation on \(A\) that is:}
\setcounter{SubsectionCounter}{1}
\textbf{\arabic{ProblemCounter}.\alph{SubsectionCounter}}
  antisymmetric and symmetric\\
  Let \(R = \{{(a,a),{(b,b)},{(c,c)}}\}\)\\
  \textit{Antisymmetric}, because for all \(x,y \in A\), \(xRy\) and \(yRx\) implies \(x=y\). This holds true as \(R\) is clearly an equivalence relation, \(a=a, b=b, c=c\)\\
  \textit{Symmetric}, because for all \(x\) and \(y\) \(\in A\), \(xRy\) and \(yRx\).\\ It is clear \(aRa \wedge aRa, bRb \wedge bRb, cRc \wedge cRc\).\\
\setcounter{SubsectionCounter}{2}
\textbf{\arabic{ProblemCounter}.\alph{SubsectionCounter}}
  antisymmetric, reflexive on \(A\), and not symmetric\\
  Let \(R = \{{(a,a),(b,b),(c,c),(a,b),{(b,c)},{(a,c)}}\}\)\\
  \textit{Reflexive}, because for all \(x \in A, xRx\)\\
  \textit{Not symmetric}, because for all \(x\) and \(y\) \(\in A\), there exists an \(xRy\) but \(y\cancel{R}x\). In this case, the pair {(a,b)} exists but not {(b,a)}.\\
  \textit{Antisymetric}, because for all \(x,y \in A\), if \(xRy\) and \(y R x\), \(x=y\). In this case, {(a,b)} exists, but (b,a) does not exist. Therefore as \(y \cancel{R} x\) for all \(x,y \in A\), the antecedent is false and thus the statement is 
  true.\\
\newpage
  \setcounter{SubsectionCounter}{3}
  \noindent\textbf{\arabic{ProblemCounter}.\alph{SubsectionCounter}}
  antisymmetric, not reflexive on \(A\), and not symmetric\\
     Let \(R = \{{(a,b),{(b,c)},{(a,c)}}\}\)\\
     \textit{Not reflexive}, because for all \(x \in A, x \cancel{R} x\). For example, {(a,a)} does not exist.\\
\textit{Antisymetric}, because for all \(x,y \in A\), if \(xRy\) and \(y R x\), \(x=y\). In this case, {(a,b)} exists, but (b,a) does not exist. Therefore as \(y \cancel{R} x\) for all \(x,y \in A\), the antecedent is false and thus the statement is 
true.\\
  \textit{Not symmetric}, because for all \(x\) and \(y\) \(\in A\), there exists an \(xRy\) but \(y\cancel{R}x\). In this case, the pair {(a,b)} exists but not 
  {(b,a)}\\
  \setcounter{SubsectionCounter}{4}
\textbf{\arabic{ProblemCounter}.\alph{SubsectionCounter}}
  symmetric and not antisymmetric\\
     Let \(R = \{{(a,b),{(b,a)}, {(a,c)}, {(c,a)}, {(b,c)}, {(c,b)}\}\)\\
     Establish \(R\) is \textit{not reflexive}, because for all \(x \in A, x \cancel{R} x\). For example, {(a,a)} does not exist.\\
         \(R\) is \textit{Symmetric}, because for all \(x\) and \(y\) \(\in A\), there exists an \(xRy\) but \(y\cancel{R}x\). In this case, the pairs {(a,b)} and {(b,a)} 
  exists, the pairs {(a,c)} and {(c,a)} exist, the pairs {(b,c)} and {(c,b)} 
  exist.\\
\(R\) is \textit{not antisymetric}. \(R\) is antisymmetric iff for all \(x,y \in A\) if \(xRy\) and \(y R x\), \(x=y\). In this case, {(a,b)} exists and {(b,a)} exists, but it is not the case that \(a = b\). This is because \(a \neq b\) as set elements must be distinct. As the antecedent of our antisymmetric definition is true, but the consequent is false, the statement that  
\(R\) is antisymmetric is therefore false. Thus \(R\) is not antisymmetric.\\
\newpage
  \setcounter{SubsectionCounter}{5}
\noindent\textbf{\arabic{ProblemCounter}.\alph{SubsectionCounter}}
  not symmetric and not antisymmetric\\
     Let \(R = \{{(a,b),{(b,a)},{(a,c)}}\}\)\\
     \textit{Not reflexive}, because for all \(x \in A, x \cancel{R} x\). For example, {(a,a)} does not exist.\\
    \textit{Not symmetric}, because for all \(x\) and \(y\) \(\in A\), there exists an \(xRy\) but \(y\cancel{R}x\) for all \(x,y \in A\). In this case, the pair {(a,c)} exists but not 
  {(c,a)}\\
\(R\) is \textit{not antisymetric}. \(R\) is antisymmetric iff for all \(x,y \in A\) if \(xRy\) and \(y R x\), \(x=y\). In this case, {(a,b)} exists and {(b,a)} exists, but it is not the case that \(a = b\). This is because \(a \neq b\) as set elements must be distinct. As the antecedent of our antisymmetric definition is true, but the consequent is false, the statement that  
\(R\) is antisymmetric is therefore false. Thus \(R\) is not antisymmetric.\\
\newpage
\setcounter{ProblemCounter}{11}
\section*{\S 3.4 Exercise \arabic{ProblemCounter}: Use your own judgment about which tasks should precede others to draw a Hasse
diagram for the partial order among the tasks for each of the following projects. :}
\setcounter{SubsectionCounter}{2}
\textbf{\arabic{ProblemCounter}.\alph{SubsectionCounter}}
To back a car out of the garage, Kim must perform 11 tasks:\\
\(t_1\): put the key in the ignition,
\(t_2\): step on the gas,
\(t_3\): check to see if the driveway is clear,
\(t_4\): start the car,
\(t_5\): adjust the mirror,
\(t_6\): open the garage door,
\(t_7\): fasten the seat belt,
\(t_8\): adjust the position of the driver’s seat,
\(t_9\): get in the car,
\(t_{10}\): put the car in reverse gear,
\(t_{11}\): step on the brake\\
\begin{tikzpicture}[scale=.7]
  \node (t_1) at (0,0) {$t_1$};
  \node (t_2) at (0,10) {$t_2$};
  \node (t_3) at (0,6) {$t_3$};
  \node (t_4) at (0,4) {$t_4$};
  \node (t_5) at (3,0) {$t_5$};
  \node (t_6) at (0,-6) {$t_6$ {(manually opened)}};
  \node (t_7) at (-3,0) {$t_7$};
  \node (t_8) at (0,-4) {$t_8$ {(I adjust before entering)}};
  \node (t_9) at (0,-2) {$t_9$};
  \node (t_10) at (0,8) {$t_{10}$};
  \node (t_11) at (0,2) {$t_{11}$};
  \draw (t_6) -- (t_8) -- (t_9) -- (t_5) -- (t_9) -- (t_1) -- (t_9) -- (t_5) -- (t_9) -- (t_7) -- (t_11) -- (t_1) -- (t_11) -- (t_5) -- (t_11) -- (t_4) -- (t_3) -- (t_10) -- (t_2);
\end{tikzpicture}\\
\newpage
\setcounter{ProblemCounter}{14}
\section*{\S 3.4 Exercise \arabic{ProblemCounter}: Let \(A\) be a set and \(\subseteq\) be the ordering for \(\mathscr{P}{(A)}\)}
\setcounter{SubsectionCounter}{1}
\textbf{\arabic{ProblemCounter}.\alph{SubsectionCounter}} Let \(C\) and \(D\) be 
subsets of \(A\). Prove that the least upper bound of \(\{C, D\}\) is \(C \cup D\) and the 
greatest lower bound of \(\{C, D\}\) is \(C \cap D\).
\begin{theorem1}
  If \(A\) is a set, if \(\subseteq\) is a ordering for \(\mathscr{P}{(A)}\), and if \(C\) and \(D\) are subsets of \(A\), then the least upper-bound 
  of \(\{C,D\}\) is \(C \cup D\) and the greatest lower bound of \(\{C,D\}\) is \(C \cap 
  D\).
  \begin{proof}
    Suppose an arbitrary \(A\) is a set and \(C\) and \(D\) are subsets of \(A\).\\
    Then \(\{C,D\} \subseteq \mathscr{P}{(A)}\).\\
    \(C \cup D\) is the least upper bound for \(\{C,D\}\) iff:
    \begin{enumerate}[(i)]
      \item \(C \cup D\) is an upper bound for \(\{C,D\}\).\\
      Let an arbitrary \(X \in \{C,D\}\). \(C \cup D\) is an upper bound for \(\{C,D\}\) iff \(X \subseteq C \cup D\) for every \(X \in \{C,D\}\). By theorem 2.2.1a, \(C \subseteq C \cup D\) 
      and \(D \subseteq C \cup D\). Therefore by definition, as an \(X \in \{C,D\}\), \(C \cup D\) is an upper bound for 
      \(\{C,D\}\).
      \item \(C \cup D \subseteq Y\) for every upper bound \(Y\) for 
      \(\{C,D\}\).\\
      Let an arbitrary \(Y\) be an upper bound for \(\{C,D\}\). By definition of an 
      upper bound, \(Y\) is an upper bound for \{C,D\} iff \(X \subseteq Y\) for every \(X \in \{C,D\}\). 
      Therefore, \(C \subseteq Y\) and \(D \subseteq Y\). Therefore \(C \cup D \subseteq 
      Y\). Therefore as an arbitrary \(Y\) is an upper bound for \(\{C,D\}\), \(C \cup D \subseteq Y\) 
      for every upper bound \(Y\) for \(\{C,D\}\).
    \end{enumerate}
    Therefore if \(A\) is a set, if \(\subseteq\) is a ordering for \(\mathscr{P}{(A)}\), and if \(C\) and \(D\) are subsets of \(A\), then the least upper-bound 
  of \(\{C,D\}\) is \(C \cup D\).
  We now show that the greatest lower bound of \(\{C,D\}\) is \(C \cap 
  D\).
  \newpage
      \noindent\(C \cap D\) is the greatest lower bound for \(\{C,D\}\) iff:
    \begin{enumerate}[(i)]
      \item \(C \cap D\) is an lower bound for \(\{C,D\}\).\\
      Let an arbitrary \(X \in \{C,D\}\). \(C \cap D\) is a lower bound for \(\{C,D\}\) iff \(C \cap D \subseteq X\) for every \(X \in \{C,D\}\). By theorem 2.2.1b, \(C \cap D \subseteq C\) and \(C \cap D \subseteq D\) hold for all sets \(C\) and \(D\). 
      Therefore by definition, as \(X \in \{C,D\}, C \cap D\) is a lower bound 
      for \(\{C,D\}\).
      \item \(Y \subseteq C \cap D\) for every lower bound \(Y\) for 
      \(\{C,D\}\).\\
      Let an arbitrary \(Y\) be a lower bound for \(\{C,D\}\). By definition of a 
      lower bound, Y is an lower bound for \{C,D\} iff \(Y \subseteq X\) for every \(X \in \{C,D\}\). 
      Therefore, \(Y \subseteq C\) and \(Y \subseteq D\). Therefore \(Y \subseteq 
      C \cap D\). Therefore as an arbitrary \(Y\) is a lower bound for \(\{C,D\}\), \(Y \subseteq C \cap D\) 
      for every lower bound \(Y\) for \(\{C,D\}\).
    \end{enumerate}
    Therefore if \(A\) is a set, if \(\subseteq\) is a ordering for \(\mathscr{P}{(A)}\), and if \(C\) and \(D\) are subsets of \(A\), then the greatest lower bound 
  of \(\{C,D\}\) is \(C \cap D\).\\
  We showed we showed if an arbitrary \(A\) is a set, if \(\subseteq\) is a ordering for \(\mathscr{P}{(A)}\), and if \(C\) and \(D\) are subsets of \(A\), then the least upper-bound 
  of \(\{C,D\}\) is \(C \cup D\) and the greatest lower bound of \(\{C,D\}\) is \(C \cap 
  D\).
  \end{proof}
\end{theorem1}
\newpage
\setcounter{ProblemCounter}{14}
\section*{\S 3.4 Exercise \arabic{ProblemCounter}: Let \(A\) be a set and \(\subseteq\) be the ordering for \(\mathscr{P}{(A)}\)}
\setcounter{SubsectionCounter}{2}
\textbf{\arabic{ProblemCounter}.\alph{SubsectionCounter}} Let \(\mathscr{P}\) be 
a family of subsets of \(A\). Prove that the least upper bound of \(\mathscr{P}\) 
is \(\bigcup\limits_{B \in \mathscr{P}}B\) and the greatest lower bound of \(\mathscr{P}\) 
is \(\bigcap\limits_{B \in \mathscr{P}}B\).
\begin{theorem1}
  If \(A\) is a set, if \(\subseteq\) is a ordering for \(\mathscr{P}{(A)}\), and if \(\mathscr{P}\) is a family of subsets of \(A\), then the least upper-bound 
  of \(\mathscr{P}\) is \(\bigcup\limits_{B \in \mathscr{P}}B\) and the greatest lower bound of \(\mathscr{P}\) is \(\bigcap\limits_{B \in \mathscr{P}}B\).
  \begin{proof}
    Suppose an arbitrary \(A\) is a set and \(\mathscr{P}\) is a family of subsets of \(A\).\\
    Then \(\mathscr{P} \subseteq \mathscr{P}{(A)}\).\\
    \(\bigcup\limits_{B \in \mathscr{P}}B\) is the least upper bound for \(\mathscr{P}\) iff:
    \begin{enumerate}[(i)]
      \item \(\bigcup\limits_{B \in \mathscr{P}}B\) is an upper bound for \(\mathscr{P}\).\\
      Let an arbitrary \(B \in \mathscr{P}\). \(\bigcup\limits_{B \in \mathscr{P}}B\) is an upper bound for \(\mathscr{P}\) iff \(B \subseteq \bigcup\limits_{B \in \mathscr{P}}B\) for every \(B \in \mathscr{P}\). By theorem 2.3.1b, \(B \subseteq \bigcup\limits_{B \in \mathscr{P}}B\). Therefore by definition, as an arbitrary \(B \in \mathscr{P}\), \(\bigcup\limits_{B \in \mathscr{P}}B\) is an upper bound for 
      \(\mathscr{P}\).
      \item \(\bigcup\limits_{B \in \mathscr{P}}B \subseteq Y\) for every upper bound \(Y\) for 
      \(\mathscr{P}\).\\
      Let an arbitrary \(Y\) be an upper bound for \(\mathscr{P}\). By definition of an 
      upper bound, \(Y\) is an upper bound for \(\mathscr{P}\) iff \(B \subseteq Y\) for every \(B \in \mathscr{P}\). 
      Therefore, \(B \subseteq Y\). Recall the proof established in \S 2.3 exercise 10c {(If \(A \subseteq D\) for every \(A \in \mathscr{A}\), then \(\bigcup\limits_{A \in \mathscr{A}}A \subseteq D\))}. Therefore \(\bigcup\limits_{B \in \mathscr{P}}B \subseteq 
      Y\). Therefore as an arbitrary \(Y\) is an upper bound for \(\mathscr{P}\), \(\bigcup\limits_{A \in \mathscr{A}}A \subseteq Y\) 
      for every upper bound \(Y\) for \(\mathscr{P}\).
    \end{enumerate}
    Therefore if \(A\) is a set, if \(\subseteq\) is a ordering for \(\mathscr{P}{(A)}\), and if \(\mathscr{P}\) are subsets of \(A\), then the least upper-bound 
  of \(\mathscr{P}\) is \(\bigcup\limits_{B \in \mathscr{P}}B\).
  We now show that the greatest lower bound of \(\mathscr{P}\) is \(\bigcap\limits_{B \in \mathscr{P}}B\).
      \noindent\(\bigcap\limits_{B \in \mathscr{P}}B\) is the greatest lower bound for \(\mathscr{P}\) iff:
    \begin{enumerate}[(i)]
      \item \(\bigcap\limits_{B \in \mathscr{P}}B\) is an lower bound for \(\mathscr{P}\).\\
      Let an arbitrary \(B \in \mathscr{P}\). \(\bigcap\limits_{B \in \mathscr{P}}B\) is a lower bound for \(\mathscr{P}\) iff \(\bigcap\limits_{B \in \mathscr{P}}B \subseteq B\) for every \(B \in \mathscr{P}\). By theorem 2.3.1a, \(\bigcap\limits_{B \in \mathscr{P}}B \subseteq B\). 
      Therefore by definition, as \(B \in \mathscr{P}, \bigcap\limits_{B \in \mathscr{P}}B\) is a lower bound 
      for \(\mathscr{P}\).
      \item \(Y \subseteq \bigcap\limits_{B \in \mathscr{P}}B\) for every lower bound \(Y\) for 
      \(\mathscr{P}\).\\
      Let an arbitrary \(Y\) be a lower bound for \(\mathscr{P}\). By definition of a 
      lower bound, Y is an lower bound for \(\mathscr{P}\) iff \(Y \subseteq B\) for every \(B \in \mathscr{P}\). 
      Recall the proof established in \S 2.3 exercise 10a {(If \(B \subseteq A\) for every \(A \in \mathscr{A}\), then \(B \subseteq \bigcap\limits_{A \in \mathscr{A}}A\))}. Therefore \(Y \subseteq 
      \bigcap\limits_{B \in \mathscr{P}}B\). Therefore as an arbitrary \(Y\) is a lower bound for \(\mathscr{P}\), \(Y \subseteq \bigcap\limits_{B \in \mathscr{P}}B\) 
      for every lower bound \(Y\) for \(\mathscr{P}\).
    \end{enumerate}
    Therefore if \(A\) is a set, if \(\subseteq\) is a ordering for \(\mathscr{P}{(A)}\), and if \(\mathscr{P}\) are subsets of \(A\), then the greatest lower bound 
  of \(\mathscr{P}\) is \(\bigcap\limits_{B \in \mathscr{P}}B\).\\
  We showed we showed if an arbitrary \(A\) is a set, if \(\subseteq\) is a ordering for \(\mathscr{P}{(A)}\), and if \(\mathscr{P}\) is a family of subsets of \(A\), then the least upper-bound 
  of \(\mathscr{P}\) is \(\bigcup\limits_{B \in \mathscr{P}}B\) and the greatest lower bound of \(\mathscr{P}\) is \(\bigcap\limits_{B \in \mathscr{P}}B\).
  \end{proof}
\end{theorem1}
\newpage
\setcounter{ProblemCounter}{18}
\section*{\S 3.4 Exercise \arabic{ProblemCounter}: Prove that every subset of a well-ordered set is well-ordered.}
\begin{theorem1}
  Every subset of a well-ordered set is well-ordered.
  \begin{proof}
    Let \(A\) be a well-ordered set.\\
    Let an arbitrary \(B \subseteq A\).\\
    Let a non-empty \(C \subseteq B\).\\
    Therefore \(C \subseteq A\) by subset transitivity.\\
    Because \(A\) is a well-ordered set, \(C\) has the smallest element.\\
    Therefore, by definition of a well-ordered set, \(B\) is well-ordered.\\
    Because \(A\) is a well-ordered set and we let an arbitrary \(B \subseteq A\) and a non-empty, \(C \subseteq B\), 
    we conclude any subset of a well-ordered set is well-ordered.
  \end{proof}
\end{theorem1}
\newpage
\setcounter{ProblemCounter}{0}
\section*{\S 3.4 Exercise \arabic{ProblemCounter}: Let \(R\) be a partial order for a set \(A\) and \(B \subseteq A\). Then if inf{(B)} exists, it is unique.}
\begin{theorem1}
  If \(R\) is a partial order for a set \(A\) and \(B \subseteq A\) and if inf{(B)} exists, then it is unique.
  \begin{proof}
    Suppose that \(x\) and \(y\) are both infima for \(B\).\\
    Then \(x\) is a lower bound for \(B\) and \(z R x\) for every lower bound \(z\) 
    for B. In particular, \(yRx\).\\
    And \(y\) is a lower bound for \(B\) and \(z R y\) for every lower bound \(z\) 
    for B. In particular, \(xRy\).\\
    If the relations \(zRx\) and \(zRy\) implies x = y, then inf{(B)} is unique.\\
    Because we let \(x\) and \(y\) be infima for B, we showed that if \(R\) is a 
    partial order for a set \(A\) and \(B \subseteq A\), then inf(B) is unique.
  \end{proof}
\end{theorem1}
\end{document}