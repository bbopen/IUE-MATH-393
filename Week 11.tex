\documentclass[a4paper,11pt]{article}
\usepackage{amsmath,amsthm,amssymb}
\usepackage{mathtools}
\usepackage{mathrsfs}
\usepackage{setspace}
\usepackage{enumerate}
\usepackage{caption}
\usepackage{pgfplots}
\DeclarePairedDelimiter\abs{\lvert}{\rvert}
\PassOptionsToPackage{usenames,dvipsnames,svgnames}{xcolor}  
\usepackage{tikz}
\usepackage{cancel}
\usetikzlibrary{arrows,positioning,automata,fit,shapes,backgrounds}
\usepackage{framed}
\pgfplotsset{every axis/.append style={
    axis x line=middle,    % put the x axis in the middle
    axis y line=middle,    % put the y axis in the middle
    axis line style={<->}, % arrows on the axis
    xlabel={$x$},          % default put x on x-axis
    ylabel={$y$},          % default put y on y-axis
    },
    cmhplot/.style={color=blue,mark=none,line width=1pt,<->},
    soldot/.style={color=blue,only marks,mark=*},
    holdot/.style={color=blue,fill=white,only marks,mark=*},
}
\begin{document}
\newtheorem*{theorem1}{Theorem}
\newtheorem*{theorem2}{Theorem}
\newtheorem*{theorem3}{Theorem}
\newtheorem*{theorem4}{Theorem}
\newtheorem*{theorem5}{Theorem}
\newtheorem*{theorem6}{Theorem}
\newtheorem*{theorem7}{Theorem}
\newtheorem*{theorem8}{Theorem}
\newtheorem*{theorem9}{True/False?}
\title{MATH 393 Week 11 Assignment 2\textsuperscript{nd} Draft}
\author{Brett Bonner}
\date{April 5, 2014}
\maketitle
\doublespacing
\newcounter{ProblemCounter}
\newcounter{SubsectionCounter}[ProblemCounter]

\setcounter{ProblemCounter}{1}
\section*{\S 4.1 Exercise \arabic{ProblemCounter}: Show that the following relations are not functions on \(\mathbb{R}\):}
\setcounter{SubsectionCounter}{1}
\textbf{\arabic{ProblemCounter}.\alph{SubsectionCounter}}
The relation \(\{{(x,y)}\in \mathbb{R}\times \mathbb{R}: x^{2} = y^{2}\}\) is not a function on \(\mathbb{R}\).
\begin{theorem1}
  \(\{{(x,y)}\in \mathbb{R}\times \mathbb{R}: x^{2} = y^{2}\}\) is not a 
  function on  \(\mathbb{R}\).
  \begin{proof}
  The relation, call it \(f\), is not a function on \(\mathbb{R}\) as \(f\) 
  contains, for example, \(\{{(0,0)}, \ldots {(1,-1), \ldots {(1,1)}, \ldots {(2,2)}, \ldots 
  {(2,-2)}}\}\). If \(f\) were a function, then by definition any \({(x,y)}\in f\) and an \({(x,z)} \in f\) would mean 
  \(y=z\). If we select pairs \({(1,-1)}\) and \({(1,1)}\) from our relation, we 
  see clearly that \(-1 \neq 1\) and therefore \(y \neq z\), thus \(f\) is not a 
  function.
   \end{proof}
  \end{theorem1}
  \setcounter{SubsectionCounter}{4}
\noindent\textbf{\arabic{ProblemCounter}.\alph{SubsectionCounter}}
\begin{theorem1}
  The relation \(\{{(x,y)}\in \mathbb{R}\times \mathbb{R}: y^{2} = \sqrt{x}\}\) is 
not a function on \{R\}.
\begin{proof}
  The relation, call it \(f\), is not a function on \(\mathbb{R}\) as \(f\) 
  contains, for example, \(\{\ldots {(0,0), \ldots {(1,1)}, \ldots {(1,-1)}, \ldots {(4,\sqrt{2})}, \ldots 
  {(16,2)}}\ldots\}\). If \(f\) were a function, then by definition any \({(x,y)}\in f\) and an \({(x,z)} \in f\) would mean 
  \(y=z\). If we select pairs \({(1,-1)}\) and \({(1,1)}\) from our relation, we 
  see clearly that \(-1 \neq 1\) and therefore \(y \neq z\), thus \(f\) is not a 
  function.
\end{proof}
\end{theorem1}
\newpage
\noindent\setcounter{ProblemCounter}{7}
\section*{\S 4.1 Exercise \arabic{ProblemCounter}: Let the universe be \(\mathbb{R}\) and \(A=[1,3{)}\). Sketch the graph of:}
\setcounter{SubsectionCounter}{1}
\textbf{\arabic{ProblemCounter}.\alph{SubsectionCounter}} \(\mathcal{X}_{A}.\)\\
\begin{tikzpicture}
    \begin{axis}[
            xmin=-3.5,xmax=5.5,
            ymin=-2,ymax=2,
            xtick={-3,-2,-1,0,1,2,3},
            ytick={-2,-1,0,1,2,3}
        ]
        \addplot[cmhplot,<-,domain=-3:1]{0};
        \addplot[cmhplot,-,domain=1:3]{1};
        \addplot[cmhplot,->,domain=3:5]{0};
        \addplot[holdot]coordinates{(1,0)(3,1)};
        \addplot[soldot]coordinates{(1,1)(3,0)};
    \end{axis}
\end{tikzpicture}\\
\setcounter{SubsectionCounter}{2}
\textbf{\arabic{ProblemCounter}.\alph{SubsectionCounter}} \(\mathcal{X}_{A^{C}}.\)\\
\begin{tikzpicture}
    \begin{axis}[
            xmin=-3.5,xmax=5.5,
            ymin=-2,ymax=2,
            xtick={-3,-2,-1,0,1,2,3},
            ytick={-2,-1,0,1,2,3}
        ]
        \addplot[cmhplot,<-,domain=-3:1]{1};
        \addplot[cmhplot,-,domain=1:3]{0};
        \addplot[cmhplot,->,domain=3:5]{1};
        \addplot[holdot]coordinates{(1,1)(3,0)};
        \addplot[soldot]coordinates{(1,0)(3,1)};
    \end{axis}
\end{tikzpicture}
\newpage
\setcounter{SubsectionCounter}{3}
\textbf{\arabic{ProblemCounter}.\alph{SubsectionCounter}} \(\mathcal{X}_{\{\frac{1}{2}\}}.\)\\
\begin{tikzpicture}
    \begin{axis}[
            xmin=-3.5,xmax=5.5,
            ymin=-2,ymax=2,
            xtick={-3,-2,-1,0,.5,1,2,3},
            xticklabels={-3, -2, -1, 0, \(\frac{1}{2}\), 1, 2, 3},
            ytick={-2,-1,0,1,2,3}
        ]
        \addplot[cmhplot,<-,domain=-3:.4]{0};
        \addplot[cmhplot,-,domain=.4:3]{0};
        \addplot[cmhplot,->,domain=.6:5]{0};
        \addplot[holdot]coordinates{(.5,0)};
        \addplot[soldot]coordinates{(.5,1)};
    \end{axis}
\end{tikzpicture}\\
\setcounter{SubsectionCounter}{4}
\textbf{\arabic{ProblemCounter}.\alph{SubsectionCounter}} \(\mathcal{X}_{\mathbb{N}}.\)\\
\begin{tikzpicture}
    \begin{axis}[
            xmin=-3.5,xmax=5.5,
            ymin=-2,ymax=2,
            xtick={-3,-2,-1,0,1,2,3},
            ytick={-2,-1,0,1,2,3}
        ]
        \addplot[cmhplot,<-,domain=-3:.4]{0};
        \addplot[cmhplot,-,domain=.4:3]{0};
        \addplot[cmhplot,->,domain=.6:5]{0};
        \addplot[holdot]coordinates{(1,0)(2,0)(3,0)(4,0)};
        \addplot[soldot]coordinates{(1,1)(2,1)(3,1)(4,1)};
    \end{axis}
\end{tikzpicture}
\newpage
\setcounter{ProblemCounter}{9}
\section*{\S 4.1 Exercise \arabic{ProblemCounter}: Give an example of a sequence \(x\) such that:}
\setcounter{SubsectionCounter}{2}
\textbf{\arabic{ProblemCounter}.\alph{SubsectionCounter}} the terms of \(x\) are alternately positive and 
negative.\\
\(x_{n}={(-1)}^{n}\)\\
\setcounter{SubsectionCounter}{4}
\textbf{\arabic{ProblemCounter}.\alph{SubsectionCounter}} the range of \(x\) has exactly 3 elements.\\
\(x_{n}=n \text{ mod } 3\)
\newpage
\setcounter{ProblemCounter}{10}
\section*{\S 4.1 Exercise \arabic{ProblemCounter}: For the canonical map \(f:\mathbb{Z}\rightarrow\mathbb{Z}_{6}\) find:}
\setcounter{SubsectionCounter}{1}
\textbf{\arabic{ProblemCounter}.\alph{SubsectionCounter}} \(f{(3)}\).\\
\(f(3) = \bar{3} = \{\ldots,-9,-3,3,9,\ldots\}\)\\
\setcounter{SubsectionCounter}{2}
\textbf{\arabic{ProblemCounter}.\alph{SubsectionCounter}} the image of 6.\\
\(y = f(6) = \bar{6} = \bar{0} = \{\ldots,-12,-6,0,6,12,\ldots\}\)\\
\setcounter{SubsectionCounter}{3}
\textbf{\arabic{ProblemCounter}.\alph{SubsectionCounter}} a preimage of \(\bar{3}\).\\
\(\bar{3} = f(3)\). 3 is the preimage of \(\bar{3}\)\\
\setcounter{SubsectionCounter}{4}
\textbf{\arabic{ProblemCounter}.\alph{SubsectionCounter}} all pre-images of \(\bar{1}\).\\
\(\bar{1} = f(1)\). Preimages of \(\bar{1}\) are \(\ldots,-5,1,7,13,\ldots\)\\
\newpage
\setcounter{ProblemCounter}{14}
\section*{\S 4.1 Exercise \arabic{ProblemCounter}: Complete the proof of Theorem 4.1.1. That is, prove that if {(i)} Dom\({(f)} =\) Dom\({(g)}\) and (ii) for all \(x \in\)Dom\({(f)}\), \(f{(x)}=g{(x)}\), then \(f=g\).:}
\begin{theorem1}
  Two functions \(f\) and \(g\) are equal iff
  \begin{enumerate}[(i)]
    \item \(\textsl{Dom}{(f)} = \textsl{Dom}{(g)}\) and
    \item for all \(x \in \textsl{Dom}{(f)}, f{(x)}=g{(x)}\)
  \end{enumerate}
  \begin{proof}
\begin{align*}
  (Dom{(f)} &= Dom{(g)} \text{ and } f{(x)}=g{(x)} \text{ for all } x \in Dom{(f)}) 
  \Rightarrow {(f = g)}\\
  &\Leftrightarrow f \subseteq g \text{ and } g \subseteq f\\
  &\Leftrightarrow x \in Dom{(f)} \text{ and } x \in Dom{(g)}\\
  &\Leftrightarrow f{(x)} = y \text{ and } g{(x)} = y\\
  &\Leftrightarrow {(x,y)} \in f,g \\
\end{align*}
  \end{proof}
\end{theorem1}
\newpage
\setcounter{ProblemCounter}{6}
\section*{\S 4.2 Exercise \arabic{ProblemCounter}: Prove the remaining parts of theorem 4.2.3: if \(f:A\rightarrow B\), then \(I_{B} \circ 
f=f\)}
\begin{theorem1}
If \(f:A\rightarrow B\), then \(I_{B} \circ 
f=f\)
   \begin{proof}
Let \(f: A \rightarrow B\).\\
\(\text{Dom}{(I_{B} \circ f)} = \text{Dom}{(
f)} = A.\)\\
If \(x \in A\), then \({(I_{B} \circ f)}{(x)} = I_{B}{(f{(x)})} = f{(x)}\).\\
Therefore \(I_{B} \circ f = f\).
     \end{proof}
  \end{theorem1}
\newpage
\setcounter{ProblemCounter}{7}
\section*{\S 4.2 Exercise \arabic{ProblemCounter}: Prove the remaining parts of theorem 4.2.4: if \(f:A\rightarrow B\) with Rng\({(f)=C,}\) and if \(f^{-1}\) is a function, then \(f \circ 
f^{-1}=I_{C}\)}
\begin{theorem1}
if \(f: A \rightarrow B\) with Rng\({(f)=C,}\) and if \(f^{-1}\) is a function, then \(f \circ 
f^{-1}=I_{C}\).
   \begin{proof}
Suppose \(f:A \rightarrow B\) is a function with \(\text{Rng}{(f)}=C\) and \(f^{-1}\) is a function.\\
Then \(\text{Dom}{(f \circ f^{-1})} = \text{Dom}{(f^{-1})}\) by Theorem 4.2.3.\\
As \(\text{Rng}{(f)} = C\), then \(f^{-1}:C \rightarrow A\) therefore \(\text{Dom}{(f^-1)} = 
C\).\\
\(Dom{(I_{C})} = C\), since \(I_{C}: C \rightarrow C\) by definition of 
identity.\\
Therefore as \(\text{Dom}{(f^{-1})} = C\) and \(\text{Dom}{(I_C)} = C\), then \(\text{Dom}{(f \circ f^{-1})} = 
\text{Dom}{(I_C)}\).\\
Let \(x \in \text{Dom}{(f^{-1})}\) be arbitrary.\\
If \(f^{-1}{(x)} = z\), then \({(x,z)} \in f^{-1}\). Therefore \({(z,x)} \in f\) 
and \(f{(z)} = x\).\\
Since \({(f \circ f^{-1})}{(x)} = f{(f^{-1}){(x)}} = f{(z)} = x\) and \(I_{C}{(x)} = 
x\),\\
Then \({(f \circ f^{-1})}{(x)} = I_{C}{(x)}\) for all \(x \in 
\text{Dom}{(f^{-1})}\).\\
Therefore, by Theorem 4.1.1, \(f \circ f^{-1} = I_{C}\). 
     \end{proof}
  \end{theorem1}
  \newpage
\setcounter{ProblemCounter}{9}
\section*{\S 4.1 Exercise \arabic{ProblemCounter}: Describe two extensions of \(f\) with the domain \(\mathbb{R}\) for the function:}
\setcounter{SubsectionCounter}{1}
\textbf{\arabic{ProblemCounter}.\alph{SubsectionCounter}} \(f = \{{(x,y)} \in \mathbb{N}\times \mathbb{N}:y=x^{2}\}\).\\
\(\{{(x,y)} \in \mathbb{R}\times\mathbb{N}:y=0 \text{ if } x < 0 \text{ and } y = x^2 \text{ if } x \geq 0\}\)\\
\(\{{(x,y)} \in \mathbb{R}\times\mathbb{R}:y=-x^{2} \text{ if } x < 0 \text{ and } y = x^2 \text{ if } x \geq 0\}\)\\
\setcounter{SubsectionCounter}{2}
\textbf{\arabic{ProblemCounter}.\alph{SubsectionCounter}} \(f = \{{(x,y)} \in \mathbb{N}\times \mathbb{N}:y=3\}\).\\
\(\{{(x,y)} \in \mathbb{R}\times\mathbb{N}:y=0 \text{ if } x < 0 \text{ and } y = 3 \text{ if } x \geq 0\}\)\\
\(\{{(x,y)} \in \mathbb{R}\times\mathbb{N}:y=0 \text{ if } x < 1 \text{ and } y = 3 \text{ if } x \geq 1\}\)\\
\setcounter{SubsectionCounter}{3}
\textbf{\arabic{ProblemCounter}.\alph{SubsectionCounter}} \(f = \{{(x,y)} \in {[-1,1]}\times {[-1,1]}:y=-x\}\).\\
\(\{{(x,y)} \in \mathbb{R}\times {[-1,1]}:y=-x \}\)\\
\(\{{(x,y)} \in \mathbb{R}\times {[-1,1]}:y=1 \text{ if } x < -1 \text{ and }y=-1 \text{ if } x > 1 \text{ and } y=-x \text{ if } -1 \leq x \leq 1\}\)\\
\newpage
\setcounter{ProblemCounter}{10}
\section*{\S 4.2 Exercise \arabic{ProblemCounter}: Prove that, if \(f\) and \(g\) are functions, then \(f \cap g\) is a function by showing that \(f \cap g = g|_{A}\) where \(A=\{x:g{(x)} = 
f{(x)}\}\)}
\begin{theorem1}
If \(f\) and \(g\) are functions such that \(f \cap g = g|_{A}\) where \(A=\{x:g{(x)} = 
f{(x)}\}\), then \(f \cap g\).
\begin{proof}
\begin{align*}
  &f \cap g = g|_{A} \text{ where } A = {x: g{(x)} = f{(x)}} \Rightarrow f \cap 
  g\\
  &\Leftrightarrow Dom{(f \cap g)} = Dom{(g|_{A})}\\
  &\Leftrightarrow Dom{(f \cap g)} \subseteq Dom{(g|_{A})} \text{ and } Dom{(g|A)} \subseteq Dom{(f \cap 
  g)}\\
  &\Leftrightarrow {(\forall x \in Dom{(f \cap g)})} {(f \cap g)}{(x)} = {(g|_{A})}{(x)}\\
  &\Leftrightarrow x \in Dom{(f \cap g)} \text{ and } x \in Dom{(g|A)}\\
  &\Leftrightarrow {(x,y)} \in f \cap g \text{ and } {(x,y)} \in g|_{A}\\
  &\Leftrightarrow {(x,y)} \in f \text{ and } {(x,y)} \in g\\
  &\Leftrightarrow y=f{(x)}=g{(x)}\\
  &\Leftrightarrow Dom{(g|_{A})} = A\\
  &\Leftrightarrow x \in A
 \end{align*}
\end{proof}
\end{theorem1}
\newpage
\setcounter{ProblemCounter}{18}
\section*{\S 4.2 Exercise \arabic{ProblemCounter}: Let \(f_1:\mathbb{R} \rightarrow \mathbb{R}\) and \(f_2:\mathbb{R} \rightarrow \mathbb{R}\). Define the pointwise sum \(f_1 + f_2\) and pointwise product \(f_1 \cdot f_2\) as follows:}
\begin{align*}
  {(f_1 + f_2)}{(x)} &= f_1{(x)} + f_2{(x)} \text{ for all } x \in \mathbb{R} \text{ 
  and}\\
  {(f_1 \cdot f_2)}{(x)} &= f_1{(x)} \cdot f_2{(x)} \text{ for all } x \in 
  \mathbb{R}.
\end{align*}
\setcounter{SubsectionCounter}{1}
\textbf{\arabic{ProblemCounter}.\alph{SubsectionCounter}} Prove that \(f_1 + f_2\) 
and \(f_1 \cdot f_2\) are functions with domain \(\mathbb{R}\).
\begin{theorem1}
  \(f_1 + f_2\) 
and \(f_1 \cdot f_2\) are functions with domain \(\mathbb{R}\)
  \begin{proof}
    \begin{align*}
      x &\in \text{ Dom}{(f_1 + f_2)} \text{ and } x \in \text{ Dom}{(f_1 \cdot 
      f_2)}\\
      &\Leftrightarrow {(f_1 + f_2)}{(x)} = y \text{ and } {(f_1 \cdot f_2)}{(x)} 
      = z \text{ for some } y \text{ and } z.\\
      &\Leftrightarrow x \in \text{ Dom}{(f_1)} and x \in \text{ Dom}{(f_2)}.\\
      &\Leftrightarrow x \in \mathbb{R}.
    \end{align*}
    Therefore, Dom\({(f_1 + f_2)} = \text{Dom}{(f_1 \cdot f_2)}=\mathbb{R}\).\\
    Now let \({(f_1 + f_2)}{(x)} = a\) and \({(f_1 + f_2)}{(x)} = b\).\\
    \(f_1{(x)} + f_2{(x)} = a\) and \(f_1{(x)} + f_2{(x)} = b\).\\
    As \(f_1 \in \mathbb{R}\) and \(f_2 \in \mathbb{R}\), \(a = f_1{(x)} = f_2{(x)} = 
    b\).\\
    Therefore if \({(f_1 + f_2)}{(x)} = a\) and \({(f_1 + f_2)}{(x)} = b\), then 
    \(a = b\).\\
    Similarly, if \({(f_1 \cdot f_2)}{(x)} = c\) and \({(f_1 \cdot f_2)}{(x)} = d\), then 
    \(c = d\).\\
    Thus \(f_1 + f_2\) and \(f_1 \cdot f_2\) are functions with domain 
    \(\mathbb{R}\).
  \end{proof}
\end{theorem1}
\newpage
\setcounter{SubsectionCounter}{2}
\textbf{\arabic{ProblemCounter}.\alph{SubsectionCounter}} Let \(f{(x)} = 2x+5, g{(x)} = 6 - 7x\), and \(h{(x)} = 3x^2 - 7x + 2\). 
Compute \({(f+g)}{(x)}, {(f \cdot g)}{(x)}, {(f + h)}{(x)},\) and \({(g \cdot 
h)}{(x)}\).
\begin{align}
  {(f+g)}{(x)} = f{(x)} + g{(x)} = 2x+5 + 6 - 7x = -5x + 11\\
  {(f \cdot g)}{(x)} = f{(x)} \cdot g{(x)} = {(2x+5)}{(6 - 7x)} = -14 x^2-23 
  x+30\\
  {(f+h)}{(x)} = f{(x)} + h{(x)} = {(2x+5)} + {(3x^2 - 7x + 2)} = 3 x^2 -5x+7\\
  {(g \cdot h)}{(x)} = g{(x)} \cdot h{(x)} = {(6 - 7x)} \cdot {(3x^2 - 7x + 2)} = -21x^3 + 67x^2 -56x + 12
\end{align}
\end{document}