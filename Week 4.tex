\documentclass[a4paper,11pt]{article}
\usepackage{amsmath,amsthm,amssymb}
\usepackage{mathtools}
\DeclarePairedDelimiter\abs{\lvert}{\rvert}
\begin{document}
\newtheorem*{theorem1}{Theorem}
\newtheorem*{theorem2}{Theorem}
\newtheorem*{theorem3}{Theorem}
\newtheorem*{theorem4}{Theorem}
\newtheorem*{theorem5}{Theorem}
\newtheorem*{theorem6}{Theorem}
\newtheorem*{theorem7}{Theorem}
\newtheorem*{theorem8}{Theorem}
\newtheorem*{disprove}{Disprove}
\title{MATH 303 Week 4 Assignment 2\textsuperscript{nd} Draft}
\author{Brett G. Bonner}
\date{February 17, 2014}
\maketitle
\linespread{1.5}
\newcounter{ProblemCounter}
\newcounter{SubsectionCounter}[ProblemCounter]
\addtocounter{ProblemCounter}{1} % set them to some other numbers than 0
\addtocounter{SubsectionCounter}{2} % same
%

\section*{\S 1.6 Exercise \arabic{ProblemCounter}: Prove that}
\textbf{\arabic{ProblemCounter}.\alph{SubsectionCounter}}
There exist integers \(m\) and \(n\) such that \(15m+12n=3\)
\begin{theorem1}
There exist integers \(m\) and \(n\) such that \(15m+12n=3\)\\
Semi-formally, \((\exists m)(\exists n)((m \wedge n)\in \mathbb{Z})(15m+12n=3)\)
\begin{proof}
Let \(m\) and \(n\) be integers.\\
Let \(m=1\) and \(n=-1\).\\
With this choice:
\begin{equation*}
  15m+12n=
  15(1)+12(-1)=
  15-12=
  3
\end{equation*}
Therefore, there exist integers \(m\) and \(n\) such that \(15m+12n=3\).\\
Because \(1\) and \(-1\) are integers, we proved\\
\((\exists m)(\exists n)((m \wedge n)\in \mathbb{Z})(15m+12n=3)\)
\end{proof}
\end{theorem1}
\newpage

\addtocounter{SubsectionCounter}{2}
\section*{\S 1.6 Exercise \arabic{ProblemCounter}: Prove that}
\textbf{\arabic{ProblemCounter}.\alph{SubsectionCounter}}
There do not exist integers \(m\) and \(n\) such that \(12m+15n=1\)
\begin{theorem2}
There do not exist integers \(m\) and \(n\) such that \(12m+15n=1\)\\
Semi-formally, \((\neg\exists m)(\neg\exists n)((m \wedge n)\in \mathbb{Z})(12m+15n=1)\)
\begin{proof}
Suppose there exist two integers \(m\) and \(n\) such that \(12m+15n=1\)\\
By factoring out 3, we determine:
\begin{gather*}
12m+15n=1 \\
3(4m+5m)=1
\end{gather*}
So 3 divides 1. This is a contradiction, because 3 does not divide 1.\\
Therefore, there can not be integers \(m\) and \(n\) such that \(12m+15n=1\).\\
We proved by contradiction there do not exist integers \(m\) and \(n\) such that 
\(12m+15n=1\)
\end{proof}
\end{theorem2}

\newpage

\setcounter{ProblemCounter}{2}
\setcounter{SubsectionCounter}{1}
\section*{\S 1.6 Exercise \arabic{ProblemCounter}: Prove that for all integers \(a\), \(b\) and \(c\)}
\textbf{\arabic{ProblemCounter}.\alph{SubsectionCounter}}
if \(c\) divides \(a\) and \(c\) divides \(b\), then for all integers \(x\) and \(y\), \(c\) divides \(ax+by\)
\begin{theorem3}
If \(c\) divides \(a\) and \(c\) divides \(b\), then for all integers \(x\) and \(y\), \(c\) divides 
\(ax+by\).\\
Semi-formally \((\forall a)(\forall b)(\forall c)[(c|a)\wedge(c|b)\Rightarrow(\forall x)(\forall y)((x\wedge y)\in \mathbb{Z})(c|(ax+by))]\)
\begin{proof}
Let \(a, b,\) and \(c\) be arbitrary integers.\\
Assume \(c\) divides \(a\) and \(c\) divides \(b\), where \(c\neq 0\).\\
Let \(x\) and \(y\) be arbitrary integers.
\begin{gather*}
  a=ck \text{ for some integer } k&\\
  b=cj \text{ for some integer } j&
\end{gather*}
Thus \(ax+by=(ck)x + (cj)y = c(kx+jy)\).\\
As \(kx+jy\) (a sum of integer products) is an integer, \(ax+by=cz\) for some 
integer \(z\).\\
Therefore, we showed that if \(c\) divides \(a\) and \(c\) divides \(b\), then 
for all integers \(x\) and \(y\), \(c\) divides \(ax+by\). Because \(a, b, c\) 
were arbitrary, we proved \((\forall a)(\forall b)(\forall c)[(c|a)\wedge(c|b)\Rightarrow(\forall x)(\forall y)((x\wedge y)\in \mathbb{Z})(c|(ax+by))]\)
\end{proof}
\end{theorem3}

\newpage

\setcounter{ProblemCounter}{2}
\setcounter{SubsectionCounter}{5}
\section*{\S 1.6 Exercise \arabic{ProblemCounter}: Prove that for all integers \(a\), \(b\) and \(c\)}
\textbf{\arabic{ProblemCounter}.\alph{SubsectionCounter}}
if there exist integers \(m\) and \(n\) such that \(am+bn=1\) and \(c \neq \pm 
1\), then \(c\) does not divide \(a\) or \(c\) does not divide \(b\).
\begin{theorem3}
If there exist integers \(m\) and \(n\) such that \(am+bn=1\) and \(c \neq \pm 
1\), then \(c\) does not divide \(a\) or \(c\) does not divide \(b\).\\
\((\forall a)(\forall b)(\forall c)((a\wedge b \wedge c)\in \mathbb{Z}) \)\\
\([(\exists m)(\exists n)((m\wedge n)\in \mathbb{Z})(am+bn=1)\wedge(c\neq \pm 1)]\Rightarrow [(c \nmid a)\vee (c \nmid b)] \)
\begin{proof}
  We argue by contradiction.\\
  Suppose \((\exists m)(\exists n)((m\wedge n)\in \mathbb{Z})(am+bn=1)\wedge(c\neq \pm 1)\wedge (c \mid a)\wedge (c \mid b) 
  \)\\
  Let \(m, n\) be integers such that \(am+bn=1\)\\
  Because \(c\) divides \(a\), \(a=ck\) where \(k\) is an integer.\\
  Because \(c\) divides \(b\), \(b=cj\) where \(j\) is an integer.\\
  From \(am+bn=1\), we obtain \(ckm+cjn=1\).\\
  Factoring \(c, c(km+jn)=1.\)\\
  Because \(km+jn\) is an integer, \(c\) divides 1. So \(c=1\) or \(c=-1\). This is a contradiction, as we assumed \(c \neq \pm 1\).\\
  Therefore, we showed\\ \([(\exists m)(\exists n)((m\wedge n)\in \mathbb{Z})(am+bn=1)\wedge(c\neq \pm 1)]\Rightarrow [(c \nmid a)\vee (c \nmid b)] 
  \)\\
  Because \(a,b,c\) were arbitrary, we proved\\
  \((\forall a)(\forall b)(\forall c)((a\wedge b \wedge c)\in \mathbb{Z}) \)\\
\([(\exists m)(\exists n)((m\wedge n)\in \mathbb{Z})(am+bn=1)\wedge(c\neq \pm 1)]\Rightarrow [(c \nmid a)\vee (c \nmid b)] \)
\end{proof}
\end{theorem3}

\newpage

\setcounter{ProblemCounter}{4}
\setcounter{SubsectionCounter}{2}
\section*{\S 1.6 Exercise \arabic{ProblemCounter}: Provide either a proof or a 
counterexample for each of these statements.}
\textbf{\arabic{ProblemCounter}.\alph{SubsectionCounter}}
\((\forall x)(\exists y)(x+y=0)\)(Universe of all reals)
\begin{theorem3}
  For all real numbers \(x\), there exists a real number \(y\) such that 
  \(x+y=0\).\\
  \((\forall x)(\exists y)((x \wedge y)\in \mathbb{R})(x+y=0)\)
\begin{proof}
Let \(x\) be an arbitrary real.\\
For all reals, there exists a real \(y\) such 
that it is the additive inverse of \(x\), where \(y=-x\).\\
Therefore, \(x+y=0\).\\
Because \(x\) is an arbitrary real, we proved that there exists a \(y\) such 
that \(x+y=0\)
\end{proof}
\end{theorem3}

\setcounter{ProblemCounter}{4}
\setcounter{SubsectionCounter}{6}
\noindent\textbf{\arabic{ProblemCounter}.\alph{SubsectionCounter}}
For all positive numbers \(x, x^2-x \geq 0\)
\begin{theorem3}
  For all numbers \(x\), if \(x > 0\), then \(x^2-x \geq 0\).\\
  \((\forall x)((x > 0)\Rightarrow (x^2-x \geq 0))\)
\begin{proof}
We \textbf{disprove} by counterexample.\\
Let \(x\) be an arbitrary positive real number.\\
We solve the inequality \(x^2-x \geq 0\):
\begin{gather*}
  x^2-x \geq 0\\ x(x-1)\geq 0\therefore x \geq 1, x \leq 0
\end{gather*}
As \(x \geq 1\) or \(x \leq 0\) to satisfy the inequality, the positive range of 
\(0<x<1\) contradicts the inequality.\\
For example, let \(x=\frac{1}{2}\)
\begin{gather*}
  x^2-x = (\frac{1}{2})^2-\frac{1}{2}=-\frac{1}{4}, \text{which is less than 0}.
\end{gather*}
Because we let \(x=\frac{1}{2}, x^2-x \geq 0\) is invalid for all positive real 
numbers \(x\).
\end{proof}
\end{theorem3}

\end{document}